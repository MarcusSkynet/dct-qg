\documentclass[11pt,a4paper]{article}

% =======================================================================
% == CANONICAL UNIFIED PREAMBLE FOR THE DCT/QG SERIES (v2.0) ==
% =======================================================================

% ------------------- Document Class & Core Setup -------------------
% Recommended base class.
% \documentclass[11pt,a4paper]{article}

% ------------------- Essential Packages -------------------
% --- Font, Encoding, and Microtypography ---
\usepackage[T1]{fontenc}
\usepackage[utf8]{inputenc}
\usepackage{lmodern}      % For crisp, scalable fonts
\usepackage{microtype}    % Improves the typographic quality of the document

% --- Page Layout & Spacing ---
\usepackage{geometry}
\geometry{margin=1in} % Standard 1-inch margins
\usepackage{setspace}
\onehalfspacing        % 1.5 line spacing for readability

% --- Mathematics & Physics Packages ---
\usepackage{amsmath,amssymb,mathtools,amsthm}
\usepackage{bm}           % For bold math symbols (\bm)
\usepackage{upgreek}      % For upright Greek letters (e.g., \upmu)
\usepackage{mathrsfs}     % For script font (\mathscr)
\usepackage{siunitx}      % For professional unit typesetting (\SI, \si)
\usepackage{physics}      % For commands like \dd, \Tr, \ket, \bra, etc. (replaces some manual defs)

% --- Citations, Links, Graphics, and Notes ---
\usepackage{hyperref}
\hypersetup{colorlinks=true,linkcolor=blue,citecolor=blue,urlcolor=blue}
\usepackage{graphicx}
\usepackage{enumitem}     % For customized lists
\usepackage{orcidlink}    % For ORCID iDs
\usepackage{tcolorbox}    % For styled boxes
\usepackage{verbatim}     % For code blocks
\usepackage{listings}     % For code listings
\usepackage{csquotes}     % For context-sensitive quotation marks (\enquote)
\usepackage{todonotes}    % For To-Do notes in the margin

% --- Optional but Recommended Packages ---
\usepackage{tikz}         % For drawing figures programmaticallly

% ------------------- Theorem-like Environments -------------------
% Standardized environments for consistent labeling.
\theoremstyle{plain}
\newtheorem{theorem}{Theorem}
\newtheorem{proposition}{Proposition}
\newtheorem{lemma}{Lemma}
\theoremstyle{definition}
\newtheorem{definition}{Definition}
\newtheorem{law}{Law}
\newtheorem{tdt}{TDT} % Specific environment for TDT laws if needed
\theoremstyle{remark}
\newtheorem{remark}{Remark}

% --- Custom Boxed Environment for Scope Remarks ---
\newtcolorbox{scopebox}[1][]{
  colback=gray!5,
  colframe=gray!75!black,
  fonttitle=\bfseries,
  title={#1},
  % If no title is given by the user, default to "Scope & Open Items"
  IfValueTF={#1}{}{title={Scope \& Open Items}}
}

% =======================================================================
% == CANONICAL MACROS (Grouped by Function) ==
% =======================================================================

% ------------------- 1. DCT Foundational Concepts (Calligraphic) -------------------
\newcommand{\LL}{\mathcal{L}}          % The Ledger surface
\newcommand{\Ical}{\mathcal{I}}        % The universal curvature invariant
\newcommand{\Icrit}{\Ical_{\mathrm{crit}}} % The critical value of the invariant
\newcommand{\T}{\mathcal{T}}           % The Transdimensional Constant
\newcommand{\Jcal}{\mathcal{J}}        % The snap instrument superoperator
\newcommand{\Rcal}{\mathcal{R}}        % The local areal radius (the geometric "ruler")
\newcommand{\Acal}{\mathcal{A}}        % Dynamical/infinitesimal area (e.g., in Raychaudhuri eq. and for area ticks)

% ------------------- 2. Boundary Dynamics (Sans-Serif) -------------------
\newcommand{\Bop}{\mathsf{B}}          % The Robin boundary operator
\newcommand{\Rrefl}{\mathsf{R}}        % The reflection coefficient
\newcommand{\SBC}{S_{\text{BC}}}       % ACTION of the Boundary Condition sector
\newcommand{\Kboundary}{\mathsf{K}}    % General boundary kernel operator (contextual)

% ------------------- 3. Actions & Lagrangians -------------------
\newcommand{\Lag}{\mathscr{L}}         % Lagrangian density (script font)
\newcommand{\SL}{S_{\mathcal{L}}}      % *** ENTROPY of the Ledger ***
\newcommand{\Sledger}{S_{\text{ledger}}} % *** ACTION of the Ledger EFT ***

% ------------------- 4. Standard Physics & GR Quantities -------------------
% --- General ---
\newcommand{\lp}{\ell_{\mathrm P}}     % Planck length
\newcommand{\MP}{M_{\mathrm P}}        % Planck mass
\newcommand{\br}{\mathrm {br}}         % The irreversible payload bit
% --- Black Holes ---
\newcommand{\RS}{R_{\mathrm S}}        % Schwarzschild radius
\newcommand{\EH}{\mathrm{H}}           % Event Horizon label (e.g., A_H, S_H)
\newcommand{\rH}{r_{\mathrm H}}        % General D-dimensional horizon radius
\newcommand{\rL}{r_\LL}                % Ledger radius
\newcommand{\AL}{A(\LL)}               % Area of the ledger
\newcommand{\AH}{A_{\mathrm H}}        % Area of the horizon
\newcommand{\THaw}{T_{\mathrm H}}      % Hawking temperature
% --- GR & Geometry ---
\newcommand{\Kretsch}{K}               % Kretschmann scalar
\newcommand{\RRm}{R_{ABCD}R^{ABCD}}    % Kretschmann scalar written out
\newcommand{\thet}{\theta}             % Expansion of a null congruence
\newcommand{\sig}{\sigma_{\mu\nu}\sigma^{\mu\nu}} % Shear norm
\newcommand{\Ric}{R_{\mu\nu}k^\mu k^\nu} % Ricci focusing term for null geodesics

% ------------------- 5. Mathematical Helpers -------------------
\newcommand{\RR}{\mathbb{R}}          % Real numbers
\newcommand{\Id}{\mathbb{I}}           % Identity operator/matrix
\newcommand{\SO}{\mathrm{SO}}         % Special Orthogonal group
\newcommand{\nplus}{n_{+}}             % Future-directed outgoing null normal
\newcommand{\nminus}{n_{-}}            % Future-directed ingoing null normal
\newcommand{\Pproj}{P^{A}{}_{B}}       % The NPR projector
\newcommand{\sgn}{\operatorname{sgn}}  % Sign function

% ------------------- 6. Quantum Mechanics Helpers -------------------
\newcommand{\Hhat}{\widehat{H}}        % Hamiltonian operator
\newcommand{\Uhat}{\widehat{U}}        % Unitary evolution operator

% ------------------- 7. Dimensional Tagging Helper -------------------
% Usage: \dimtag{D}{R} produces the D-dimensional Ricci scalar.
\newcommand{\dimtag}[2]{{}^{(#1)}\!{#2}}
\newcommand{\Dtag}{(D)}                % Dimensional tag
\newcommand{\dtag}{(d)}                % D-2 Dimensional tag
\newcommand{\lDtag}{{}^{\Dtag}}        % Left dimensional tag
\newcommand{\ldtag}{{}^{\dtag}}        % Left D-2 dimensional tag

% ------------------- 8. Numerical Anchors -------------------
\newcommand{\numTval}{0.36067376}      % 1/(4 ln 2)
\newcommand{\numIcritval}{33.27106467} % 48 ln 2

% ------------------- 9. Phenomenological Modules -------------------
% --- 9a. Echoes & Scattering ---
\newcommand{\El}{\mathsf{E}_{\ell}}    % Echo transfer function
\newcommand{\tauRT}{\tau}              % Round-trip time / Echo delay parameter
\newcommand{\phase}{\varphi}           % The phase angle symbol
\newcommand{\Rreflw}{\Rrefl_{\mathrm w,\ell}} % Wall reflectivity
\newcommand{\Rreflb}{\Rrefl_{b,\ell}}  % Barrier reflectivity
\newcommand{\Tw}{\mathsf{T}_{\mathrm w,\ell}} % Wall transmissivity
\newcommand{\Tb}{\mathsf{T}_{b,\ell}}  % Barrier transmissivity
% --- 9b. Tensor Networks & QEC ---
\newcommand{\Viso}{V}                  % Node isometry
\newcommand{\Xtot}{X_{\mathrm{tot}}}   % Total X operator
\newcommand{\Ytot}{Y_{\mathrm{tot}}}   % Total Y operator
\newcommand{\Ztot}{Z_{\mathrm{tot}}}   % Total Z operator
% --- 9c. Dark Matter & Remnants ---
\newcommand{\wgap}{\omega_{\mathrm{gap}}} % Evaporation mass gap frequency
\newcommand{\Mrem}{M_{\mathrm{rem}}}   % Remnant mass
\newcommand{\gH}{\gamma_{\mathrm H}}      % Hawking effective greybody coefficient
\newcommand{\chiH}{\chi}                  % Robin suppression factor
\newcommand{\betaf}{\beta_{\!f}}          % Initial PBH fraction at formation
% --- 9d. Dark Energy & Cosmology ---
\newcommand{\Hcal}{\mathcal{H}}         % Conformal Hubble parameter
\newcommand{\Cdot}{\dot{\mathcal{C}}}   % Ledger capacity growth rate
\newcommand{\rhoL}{\rho_{\LL}}          % Ledger energy density
\newcommand{\pL}{p_{\LL}}               % Ledger pressure
\newcommand{\wL}{w_{\LL}}               % Ledger equation of state
\newcommand{\csL}{c_{s,\LL}^2}        % Ledger sound speed
\newcommand{\OM}{\Omega_{\mathrm m}}    % Matter density parameter
\newcommand{\OL}{\Omega_{\LL}}          % Ledger (DE) density parameter
\newcommand{\ORad}{\Omega_{\mathrm r}}    % Radiation density parameter
% --- 9e. Radion & Fifth Force ---
\newcommand{\MD}{M_{D}}                % D-dimensional Planck mass
\newcommand{\varphiRad}{\varphi}       % Canonical radion field
\newcommand{\mn}{m_n}                   % KK mode mass
\newcommand{\Lef}{L_{\mathrm{eff}}}    % Effective compactification length
\newcommand{\lambdaff}{\lambda_{\mathrm{5th}}} % Fifth-force range

% =======================================================================
% == DEPRECATED / DUPLICATED MACROS (For Search & Replace) ==
% =======================================================================
% These are kept here (commented out) to help you find and replace
% old commands in your documents.

% %\newcommand{\aH}{\alpha_{\mathrm H}}     % DEPRECATED. Use \T instead.
% %\newcommand{\CurvTrigger}{\Ical=\Icrit} % DEPRECATED. Use the equation directly.
% %\newcommand{\Area}{\mathcal A}          % DEPRECATED. Use \Acal.
% %\newcommand{\Bboundary}{\mathsf{B}}      % DUPLICATE of \Bop.
% %\newcommand{\Rw}{\Rrefl_{\mathrm w,\ell}}% DUPLICATE of \Rreflw.
% %\newcommand{\Rb}{\Rrefl_{b,\ell}}      % DUPLICATE of \Rreflb.
% %\newcommand{\MP}{M_{\mathrm P}}           % DUPLICATE of M_P macro.
% %\newcommand{\Mpl}{M_{\mathrm{Pl}}}      % DUPLICATE of M_P macro.
% %\newcommand{\PNPR}{\mathsf P_{\mathrm{NPR}}} % DEPRECATED. Use \Pproj.
% %\newcommand{\tRT}{\tau_{\mathrm{RT}}}   % DEPRECATED. Use \tauRT.
% %\newcommand{\omegaGap}{\omega_{\mathrm{gap}}} % DUPLICATE of \wgap.
% %\newcommand{\Deltaecho}{\Delta t_{\mathrm{echo}}} % DEPRECATED. Use \tauRT.
% %\newcommand{\lPD}{{}^{(D)}\ell_{\mathrm P}} % DEPRECATED. Use \dimtag{D}{\lp}.
% %\newcommand{\Dtag}{(D)}                 % DEPRECATED. Use \dimtag.
% %\newcommand{\dtag}{(d)}                 % DEPRECATED. Use \dimtag.
% %\newcommand{\lDtag}{{}^{\Dtag}}         % DEPRECATED. Use \dimtag.
% %\newcommand{\ldtag}{{}^{\dtag}}         % DEPRECATED. Use \dimtag.



\title{
    \bfseries On the Necessity of Codimension-2 Reduction in Resolving Spacetime Singularities
    }
    
\author{
    Marek Hubka\, \orcidlink{0009-0003-2476-9017}
    \thanks{
        Independent Researcher, Czech Republic. \\
        Website: \href{http://tidesofuncertainty.com}{tidesofuncertainty.com}.
        Email: \href{mailto:marek@tidesofuncertainty.com}{marek@tidesofuncertainty.com}.
    }
}
\date{\today}

\begin{document}
\maketitle
\vspace{-2ex}

\begin{abstract}
This article motivates the foundational axiom of Dimensional Collapse Theory
(DCT/QG): that when curvature reaches a universal threshold, spacetime undergoes
a local, codimension-2 ``snap'' which removes a null 2-plane and leaves behind
a spacelike \((D-2)\)-dimensional surface---the Ledger~\(\LL\). The aim is not
to derive this rule from deeper principles, but to show that it is a highly
constrained and natural choice once we impose a small set of physical demands:
locality, covariance, causality, boost invariance, and compatibility with
area-based thermodynamics. We review the singularity problem as a geometric
crisis driven by the Raychaudhuri equation, enumerate possible local geometric
responses at a curvature threshold, and argue that a codimension-1 or higher
codimension reduction fails to provide a stable, thermodynamic boundary. In
contrast, deleting exactly the two null directions normal to a collapsing
cross-section yields a unique ``Goldilocks'' option: it halts catastrophic
focusing, preserves causal structure, creates a natural area carrier for
entropy, and sets the stage for the detailed constructions in Papers~I--IV of
the DCTQG series.
\end{abstract}

\tableofcontents

\section{Singularities as a geometric crisis}

The classical singularity theorems show that, under broad conditions, General
Relativity predicts geodesic incompleteness and curvature blow-ups
\cite{HawkingEllis1973}. The underlying mechanism is not mysterious: it is the
relentless focusing of geodesic congruences.

For a null congruence with tangent \(k^{a}\), expansion \(\theta\), and affine
parameter \(\lambda\), the Raychaudhuri equation in \(D\) spacetime dimensions
is \cite{Raychaudhuri1955,Poisson2004}
\begin{equation}
\frac{\dd\theta}{\dd\lambda}
=
-\frac{1}{D-2}\,\theta^{2}
- \sigma_{ab}\sigma^{ab}
- R_{ab}k^{a}k^{b},
\qquad
\frac{\dd \ln \Acal}{\dd\lambda} = \theta,
\label{eq:raychaudhuri}
\end{equation}
where \(\sigma_{ab}\) is the shear and \(\Acal(\lambda)\) is the cross-sectional
area of an infinitesimal bundle. Under the null energy condition (NEC),
\(R_{ab}k^{a}k^{b}\ge 0\), so a negative \(\theta\) typically runs away to
\(-\infty\) in finite affine parameter. The area \(\Acal\) shrinks to zero and
curvature invariants diverge.

From the perspective of effective field theory, this signals a \emph{geometric
crisis}: the classical description runs out before we reach the regime we would
like to probe. A satisfactory resolution should obey at least four conditions:
\begin{enumerate}[label=(\alph*)]
\item \textbf{Locality and covariance:} the rule must be formulated locally in
terms of tensorial quantities, without reference to global coordinates or
boundary conditions.

\item \textbf{Causality:} the resolution must not introduce superluminal
signaling or retro-causal behavior.

\item \textbf{Thermodynamic compatibility:} in the presence of horizons, any
resolution should respect the area-entropy connection
\cite{Bekenstein1973,Hawking1975,Wald1993} and the generalized second law
\cite{Bousso1999}.

\item \textbf{Mildness:} the modification should be as small and surgical as
possible, leaving ordinary GR+QFT intact where curvature is subcritical.
\end{enumerate}

Dimensional Collapse Theory starts from the idea that the breakdown occurs
because we are trying to continue the geometry \emph{through} a regime in which
the classical dimensionality of spacetime is no longer the correct
description. The central proposal is that spacetime carries a simple, local
``click rule'': when a clean, dimensionless invariant
\(\Ical[g]\) crosses a universal threshold \(\Icrit\)
\cite{DCTQG01}, the manifold undergoes a small, discrete change in its
effective dimensionality.

\section{Local curvature triggers and click rules}

The first step is to ensure that the trigger is geometric and local. In DCT we
use a curvature scalar of the schematic form
\begin{equation}
\Ical \;\equiv\; K L^{4},
\qquad
K = R_{ABCD}R^{ABCD},
\end{equation}
where \(L\) is a macroscopic length scale set by the environment
\cite{DCTQG01}. The details of the choice and the derivation of the universal
critical value \(\Icrit\) are the subject of Paper~I. For present purposes we
only need three qualitative properties:
\begin{enumerate}[label=(\roman*)]
\item \(\Ical\) is scalar and local.
\item \(\Ical\) is dimensionless and insensitive to trivial rescalings.
\item There exists a universal critical value \(\Icrit\) such that, in
astrophysical black holes, \(\Ical\to\Icrit\) just inside the event horizon.
\end{enumerate}

The click rule is:
\begin{equation}
\Ical \;=\; \Icrit
\quad\Longrightarrow\quad
\text{spacetime performs a local \emph{snap}}.
\end{equation}
The rest of the DCTQG series works out ``what the snap does'' in detail. This
paper focuses on one key aspect: \emph{how many dimensions are removed} and
\emph{why}.

\section{A menu of geometric responses}

Suppose that in some small neighborhood the curvature grows and
\(\Ical\to\Icrit\). What can the geometry do, locally and covariantly, to avoid
the catastrophic focusing implied by Eq.~\eqref{eq:raychaudhuri}?

At an abstract level, there are three broad classes of response:
\begin{enumerate}[label=(\alph*)]
\item \textbf{Modify the dynamics:} change the right-hand side of the
Raychaudhuri equation by violating the energy conditions, introducing new
repulsive forces, or changing the Einstein equations themselves.

\item \textbf{Impose nonlocal structure:} introduce global matching conditions,
wormholes, or other topological identifications which redirect the geodesics
elsewhere.

\item \textbf{Modify the manifold structure:} change the dimensionality of the
effective spacetime in the region where the crisis occurs.
\end{enumerate}

Options (a) and (b) are widely explored in the literature. DCT instead pursues
option (c): keep the field equations and local energy conditions intact as long
as possible, and let spacetime itself take a small, discrete step in
dimensionality when the click rule fires. The question then becomes sharp:
\emph{what codimension \(\Delta D\) is admissible?}

\section{Why codimension one is not enough}

A seemingly mild modification is to remove a single null direction, i.e.\ to
replace a \(D\)-dimensional region by a \((D-1)\)-dimensional null hypersurface
when \(\Ical=\Icrit\). This effectively projects the dynamics onto a lightfront
and might appear to ``cap off'' the focusing.

However, codimension-one reduction fails on two fronts:

\paragraph{(1) Focusing persists within the null hypersurface.}
A null hypersurface still admits null congruences tangent to it, with their own
expansions and shears. The Raychaudhuri equation continues to drive
\(\Acal\to 0\) \emph{within} the reduced geometry. One has simply moved the
problem from the bulk to the boundary, not removed it.

\paragraph{(2) Null surfaces do not carry a robust area entropy.}
Bekenstein--Hawking entropy and the generalized entropy functionals used in
quantum extremal surface (QES) constructions are built on \((D-2)\)-dimensional
spacelike surfaces with well-defined area \cite{Bekenstein1973,Hawking1975,
Wald1993,EngelhardtWall2015}. The ``area'' of a patch on a null hypersurface
depends on an arbitrary choice of cross-section and affine parameter; it cannot
serve as a stable, frame-independent measure of thermodynamic entropy. Any
theory that aims to preserve the area--entropy connection in the spirit of
Jacobson's derivation of Einstein's equations \cite{Jacobson1995} must treat
null sheets as \emph{carriers} of flux, not as the fundamental storage surface
for entropy.

For DCT, which explicitly uses area quanta on a codimension-2 Ledger to
account for information payloads \cite{DCTQG03}, a codimension-1 reduction
would be a poor foundation.

\section{Codimension-two removal as the ``Goldilocks'' option}

DCT instead adopts a codimension-two operation: a local removal of the normal
null 2-plane spanned by \((\nplus,\nminus)\). Concretely, pick null directors
\(\nplus^{A},\nminus^{A}\) satisfying
\begin{equation}
\nplus\!\cdot\nminus=-1,
\end{equation}
and define the tangential projector
\begin{equation}
P^{A}{}_{B}
=
\delta^{A}{}_{B}
+ \nplus^{A}\nminus_{B}
+ \nminus^{A}\nplus_{B}.
\end{equation}
The Null--Pair Removal (NPR) map of Paper~II \cite{DCTQG02} contracts every
index of every tensor with \(P^{A}{}_{B}\), deleting components along the normal
null 2-plane while preserving the intrinsic \((D-2)\)-dimensional geometry.

This choice satisfies the constraints listed in Sec.~1 in a remarkably tight
way:
\begin{enumerate}[label=(\alph*)]
\item \textbf{Minimality.} Removing exactly the two null directions normal to a
collapsing cross-section is the smallest possible change that directly targets
the focusing problem in Eq.~\eqref{eq:raychaudhuri}. Higher-codimension
removals would overkill the geometry and destroy more structure than necessary.

\item \textbf{Boost invariance.} A single null direction has no invariant
normalization; rescalings \(k^{a}\to \ee^{\lambda}k^{a}\) are an exact local
symmetry. The pair \((\nplus,\nminus)\), by contrast, carries an invariant
boost class under \(\mathrm{SO}(1,1)\): \(\nplus\to \ee^{\lambda}\nplus\),
\(\nminus\to \ee^{-\lambda}\nminus\). The projector \(P^{A}{}_{B}\) is built to
be invariant under this symmetry. Any local rule that singles out one null
direction over the other would explicitly break this gauge freedom; NPR does
not.

\item \textbf{Thermodynamic canvas.} The image of \(P^{A}{}_{B}\) is a
spacelike \((D-2)\)-dimensional surface \(\LL\) with intrinsic metric
\(h_{ab}\) and area \(A(\LL)\). This provides a natural ``canvas'' for
area-based entropy. In Paper~I, this surface is placed at a universal radius
inside a Schwarzschild black hole via the curvature trigger \(\Ical=\Icrit\)
\cite{DCTQG01}, and in Paper~III its area is quantized by the rules of
Transdimensional Thermodynamics (TDT) \cite{DCTQG03}.

\item \textbf{Information accounting and reversibility.} Removing a null
2-plane is not purely topological: it deletes geometric data (orientation,
shear alignment, twist) that can, in principle, be recorded. This is precisely
what the Infinity bits \(\{W,X,Y,Z\}\) of Paper~IV are designed to capture
\cite{DCTQG04}. NPR plus these bits make the snap reversible up to smooth
gauge, in the same way that recording a small set of discrete choices restores
reversibility to a coarse-grained map.
\end{enumerate}

In this sense, codimension-two reduction is a ``Goldilocks'' operation: any
less and the singularity problem persists; any more and one loses the geometric
structure needed to define a thermodynamic boundary.

\section{How the axiom organizes the DCTQG series}

The codimension-2 snap is the only genuine axiom of DCT/QG. Everything else in
the series is derived from, or calibrated against, this geometric choice.

\begin{itemize}[leftmargin=*]
\item \textbf{Part I.} Given the click rule~\(\Ical=\Icrit\), the
location of \(\LL\) in Schwarzschild and more general black-hole geometries is
computed explicitly, and the universal constant \(\T\) is derived from the
requirement that Ledger and horizon entropies remain consistent with
Bekenstein--Hawking and QES bounds \cite{DCTQG01}.

\item \textbf{Part II.} NPR is worked out in detail: the precise
form of \(P^{A}{}_{B}\), its boost invariance, and its action on fields of
arbitrary rank are specified. The Raychaudhuri equation is updated to a form
that remains finite in the presence of snaps \cite{DCTQG02}.

\item \textbf{Part III.} Transdimensional Thermodynamics is
constructed on \(\LL\): one bit of payload entropy costs a fixed quantum of
area, and the Ledger entropy grows in a staircase fashion at snaps while
remaining constant between them \cite{DCTQG03}.

\item \textbf{Part IV.} The Infinity bits \(\{W,X,Y,Z\}\) are
introduced as a minimal, boost-invariant register that records discrete
kinematic information about the removed null 2-plane, making the NPR operation
reversible up to smooth gauge and allowing a clean information-theoretic
interpretation of snaps \cite{DCTQG04}.
\end{itemize}

Seen from this vantage point, the codimension-2 axiom is analogous to the
Schr\"odinger equation in quantum mechanics: it is not derived from something
deeper within the framework, but it proves to be the unique simple postulate
from which a large, tightly constrained structure grows.

\section{Conclusion}

The dimensional-collapse axiom of DCT/QG can be stated in one line:
\emph{when a local curvature invariant \(\Ical\) reaches a universal threshold
\(\Icrit\), spacetime undergoes a codimension-2 snap that removes the normal
null 2-plane and leaves a spacelike Ledger surface \(\LL\) behind.} This
paper has argued that, once we insist on locality, covariance, causality,
boost invariance, and compatibility with area-based thermodynamics, this is not
a whimsical choice among many, but a highly constrained option.

A codimension-1 reduction fails to halt focusing and cannot carry a stable
entropy. Higher-codimension reductions destroy too much structure. Only the
codimension-2 NPR operation both disarms the singularity mechanism and
provides a natural canvas for entropy, information accounting, and reversible
reconstruction. The success of DCT/QG as a whole---from Ledger placement and
TDT to Infinity bits and echoes phenomenology---is, in this sense, the primary
evidence in favor of this simple axiom.

\bibliographystyle{unsrt}
\bibliography{bibliography/dct_refs,bibliography/external_refs}

\end{document}