\documentclass[11pt,a4paper]{article}

% =======================================================================
% == CANONICAL UNIFIED PREAMBLE FOR THE DCT/QG SERIES (v2.0) ==
% =======================================================================

% ------------------- Document Class & Core Setup -------------------
% Recommended base class.
% \documentclass[11pt,a4paper]{article}

% ------------------- Essential Packages -------------------
% --- Font, Encoding, and Microtypography ---
\usepackage[T1]{fontenc}
\usepackage[utf8]{inputenc}
\usepackage{lmodern}      % For crisp, scalable fonts
\usepackage{microtype}    % Improves the typographic quality of the document

% --- Page Layout & Spacing ---
\usepackage{geometry}
\geometry{margin=1in} % Standard 1-inch margins
\usepackage{setspace}
\onehalfspacing        % 1.5 line spacing for readability

% --- Mathematics & Physics Packages ---
\usepackage{amsmath,amssymb,mathtools,amsthm}
\usepackage{bm}           % For bold math symbols (\bm)
\usepackage{upgreek}      % For upright Greek letters (e.g., \upmu)
\usepackage{mathrsfs}     % For script font (\mathscr)
\usepackage{siunitx}      % For professional unit typesetting (\SI, \si)
\usepackage{physics}      % For commands like \dd, \Tr, \ket, \bra, etc. (replaces some manual defs)

% --- Citations, Links, Graphics, and Notes ---
\usepackage{hyperref}
\hypersetup{colorlinks=true,linkcolor=blue,citecolor=blue,urlcolor=blue}
\usepackage{graphicx}
\usepackage{enumitem}     % For customized lists
\usepackage{orcidlink}    % For ORCID iDs
\usepackage{tcolorbox}    % For styled boxes
\usepackage{verbatim}     % For code blocks
\usepackage{listings}     % For code listings
\usepackage{csquotes}     % For context-sensitive quotation marks (\enquote)
\usepackage{todonotes}    % For To-Do notes in the margin

% --- Optional but Recommended Packages ---
\usepackage{tikz}         % For drawing figures programmaticallly

% ------------------- Theorem-like Environments -------------------
% Standardized environments for consistent labeling.
\theoremstyle{plain}
\newtheorem{theorem}{Theorem}
\newtheorem{proposition}{Proposition}
\newtheorem{lemma}{Lemma}
\theoremstyle{definition}
\newtheorem{definition}{Definition}
\newtheorem{law}{Law}
\newtheorem{tdt}{TDT} % Specific environment for TDT laws if needed
\theoremstyle{remark}
\newtheorem{remark}{Remark}

% --- Custom Boxed Environment for Scope Remarks ---
\newtcolorbox{scopebox}[1][]{
  colback=gray!5,
  colframe=gray!75!black,
  fonttitle=\bfseries,
  title={#1},
  % If no title is given by the user, default to "Scope & Open Items"
  IfValueTF={#1}{}{title={Scope \& Open Items}}
}

% =======================================================================
% == CANONICAL MACROS (Grouped by Function) ==
% =======================================================================

% ------------------- 1. DCT Foundational Concepts (Calligraphic) -------------------
\newcommand{\LL}{\mathcal{L}}          % The Ledger surface
\newcommand{\Ical}{\mathcal{I}}        % The universal curvature invariant
\newcommand{\Icrit}{\Ical_{\mathrm{crit}}} % The critical value of the invariant
\newcommand{\T}{\mathcal{T}}           % The Transdimensional Constant
\newcommand{\Jcal}{\mathcal{J}}        % The snap instrument superoperator
\newcommand{\Rcal}{\mathcal{R}}        % The local areal radius (the geometric "ruler")
\newcommand{\Acal}{\mathcal{A}}        % Dynamical/infinitesimal area (e.g., in Raychaudhuri eq. and for area ticks)

% ------------------- 2. Boundary Dynamics (Sans-Serif) -------------------
\newcommand{\Bop}{\mathsf{B}}          % The Robin boundary operator
\newcommand{\Rrefl}{\mathsf{R}}        % The reflection coefficient
\newcommand{\SBC}{S_{\text{BC}}}       % ACTION of the Boundary Condition sector
\newcommand{\Kboundary}{\mathsf{K}}    % General boundary kernel operator (contextual)

% ------------------- 3. Actions & Lagrangians -------------------
\newcommand{\Lag}{\mathscr{L}}         % Lagrangian density (script font)
\newcommand{\SL}{S_{\mathcal{L}}}      % *** ENTROPY of the Ledger ***
\newcommand{\Sledger}{S_{\text{ledger}}} % *** ACTION of the Ledger EFT ***

% ------------------- 4. Standard Physics & GR Quantities -------------------
% --- General ---
\newcommand{\lp}{\ell_{\mathrm P}}     % Planck length
\newcommand{\MP}{M_{\mathrm P}}        % Planck mass
\newcommand{\br}{\mathrm {br}}         % The irreversible payload bit
% --- Black Holes ---
\newcommand{\RS}{R_{\mathrm S}}        % Schwarzschild radius
\newcommand{\EH}{\mathrm{H}}           % Event Horizon label (e.g., A_H, S_H)
\newcommand{\rH}{r_{\mathrm H}}        % General D-dimensional horizon radius
\newcommand{\rL}{r_\LL}                % Ledger radius
\newcommand{\AL}{A(\LL)}               % Area of the ledger
\newcommand{\AH}{A_{\mathrm H}}        % Area of the horizon
\newcommand{\THaw}{T_{\mathrm H}}      % Hawking temperature
% --- GR & Geometry ---
\newcommand{\Kretsch}{K}               % Kretschmann scalar
\newcommand{\RRm}{R_{ABCD}R^{ABCD}}    % Kretschmann scalar written out
\newcommand{\thet}{\theta}             % Expansion of a null congruence
\newcommand{\sig}{\sigma_{\mu\nu}\sigma^{\mu\nu}} % Shear norm
\newcommand{\Ric}{R_{\mu\nu}k^\mu k^\nu} % Ricci focusing term for null geodesics

% ------------------- 5. Mathematical Helpers -------------------
\newcommand{\RR}{\mathbb{R}}          % Real numbers
\newcommand{\Id}{\mathbb{I}}           % Identity operator/matrix
\newcommand{\SO}{\mathrm{SO}}         % Special Orthogonal group
\newcommand{\nplus}{n_{+}}             % Future-directed outgoing null normal
\newcommand{\nminus}{n_{-}}            % Future-directed ingoing null normal
\newcommand{\Pproj}{P^{A}{}_{B}}       % The NPR projector
\newcommand{\sgn}{\operatorname{sgn}}  % Sign function

% ------------------- 6. Quantum Mechanics Helpers -------------------
\newcommand{\Hhat}{\widehat{H}}        % Hamiltonian operator
\newcommand{\Uhat}{\widehat{U}}        % Unitary evolution operator

% ------------------- 7. Dimensional Tagging Helper -------------------
% Usage: \dimtag{D}{R} produces the D-dimensional Ricci scalar.
\newcommand{\dimtag}[2]{{}^{(#1)}\!{#2}}
\newcommand{\Dtag}{(D)}                % Dimensional tag
\newcommand{\dtag}{(d)}                % D-2 Dimensional tag
\newcommand{\lDtag}{{}^{\Dtag}}        % Left dimensional tag
\newcommand{\ldtag}{{}^{\dtag}}        % Left D-2 dimensional tag

% ------------------- 8. Numerical Anchors -------------------
\newcommand{\numTval}{0.36067376}      % 1/(4 ln 2)
\newcommand{\numIcritval}{33.27106467} % 48 ln 2

% ------------------- 9. Phenomenological Modules -------------------
% --- 9a. Echoes & Scattering ---
\newcommand{\El}{\mathsf{E}_{\ell}}    % Echo transfer function
\newcommand{\tauRT}{\tau}              % Round-trip time / Echo delay parameter
\newcommand{\phase}{\varphi}           % The phase angle symbol
\newcommand{\Rreflw}{\Rrefl_{\mathrm w,\ell}} % Wall reflectivity
\newcommand{\Rreflb}{\Rrefl_{b,\ell}}  % Barrier reflectivity
\newcommand{\Tw}{\mathsf{T}_{\mathrm w,\ell}} % Wall transmissivity
\newcommand{\Tb}{\mathsf{T}_{b,\ell}}  % Barrier transmissivity
% --- 9b. Tensor Networks & QEC ---
\newcommand{\Viso}{V}                  % Node isometry
\newcommand{\Xtot}{X_{\mathrm{tot}}}   % Total X operator
\newcommand{\Ytot}{Y_{\mathrm{tot}}}   % Total Y operator
\newcommand{\Ztot}{Z_{\mathrm{tot}}}   % Total Z operator
% --- 9c. Dark Matter & Remnants ---
\newcommand{\wgap}{\omega_{\mathrm{gap}}} % Evaporation mass gap frequency
\newcommand{\Mrem}{M_{\mathrm{rem}}}   % Remnant mass
\newcommand{\gH}{\gamma_{\mathrm H}}      % Hawking effective greybody coefficient
\newcommand{\chiH}{\chi}                  % Robin suppression factor
\newcommand{\betaf}{\beta_{\!f}}          % Initial PBH fraction at formation
% --- 9d. Dark Energy & Cosmology ---
\newcommand{\Hcal}{\mathcal{H}}         % Conformal Hubble parameter
\newcommand{\Cdot}{\dot{\mathcal{C}}}   % Ledger capacity growth rate
\newcommand{\rhoL}{\rho_{\LL}}          % Ledger energy density
\newcommand{\pL}{p_{\LL}}               % Ledger pressure
\newcommand{\wL}{w_{\LL}}               % Ledger equation of state
\newcommand{\csL}{c_{s,\LL}^2}        % Ledger sound speed
\newcommand{\OM}{\Omega_{\mathrm m}}    % Matter density parameter
\newcommand{\OL}{\Omega_{\LL}}          % Ledger (DE) density parameter
\newcommand{\ORad}{\Omega_{\mathrm r}}    % Radiation density parameter
% --- 9e. Radion & Fifth Force ---
\newcommand{\MD}{M_{D}}                % D-dimensional Planck mass
\newcommand{\varphiRad}{\varphi}       % Canonical radion field
\newcommand{\mn}{m_n}                   % KK mode mass
\newcommand{\Lef}{L_{\mathrm{eff}}}    % Effective compactification length
\newcommand{\lambdaff}{\lambda_{\mathrm{5th}}} % Fifth-force range

% =======================================================================
% == DEPRECATED / DUPLICATED MACROS (For Search & Replace) ==
% =======================================================================
% These are kept here (commented out) to help you find and replace
% old commands in your documents.

% %\newcommand{\aH}{\alpha_{\mathrm H}}     % DEPRECATED. Use \T instead.
% %\newcommand{\CurvTrigger}{\Ical=\Icrit} % DEPRECATED. Use the equation directly.
% %\newcommand{\Area}{\mathcal A}          % DEPRECATED. Use \Acal.
% %\newcommand{\Bboundary}{\mathsf{B}}      % DUPLICATE of \Bop.
% %\newcommand{\Rw}{\Rrefl_{\mathrm w,\ell}}% DUPLICATE of \Rreflw.
% %\newcommand{\Rb}{\Rrefl_{b,\ell}}      % DUPLICATE of \Rreflb.
% %\newcommand{\MP}{M_{\mathrm P}}           % DUPLICATE of M_P macro.
% %\newcommand{\Mpl}{M_{\mathrm{Pl}}}      % DUPLICATE of M_P macro.
% %\newcommand{\PNPR}{\mathsf P_{\mathrm{NPR}}} % DEPRECATED. Use \Pproj.
% %\newcommand{\tRT}{\tau_{\mathrm{RT}}}   % DEPRECATED. Use \tauRT.
% %\newcommand{\omegaGap}{\omega_{\mathrm{gap}}} % DUPLICATE of \wgap.
% %\newcommand{\Deltaecho}{\Delta t_{\mathrm{echo}}} % DEPRECATED. Use \tauRT.
% %\newcommand{\lPD}{{}^{(D)}\ell_{\mathrm P}} % DEPRECATED. Use \dimtag{D}{\lp}.
% %\newcommand{\Dtag}{(D)}                 % DEPRECATED. Use \dimtag.
% %\newcommand{\dtag}{(d)}                 % DEPRECATED. Use \dimtag.
% %\newcommand{\lDtag}{{}^{\Dtag}}         % DEPRECATED. Use \dimtag.
% %\newcommand{\ldtag}{{}^{\dtag}}         % DEPRECATED. Use \dimtag.


\title{
    \bfseries The Laws of Transdimensional Thermodynamics:\\[4pt]
    \large Capacity, Payload, and the Area Quantum at the Ledger in DCT\\[8pt]
}

\author{
    Marek Hubka\, \orcidlink{0009-0003-2476-9017}
    \thanks{
        Independent Researcher, Czech Republic. \\
        Website: \href{http://www.tidesofuncertainty.com}{tidesofuncertainty.com}.
        Email: \href{mailto:marek@tidesofuncertainty.com}{marek@tidesofuncertainty.com}.
    }
}
\date{\today}

\begin{document}
\maketitle
\vspace{-2ex}

\begin{abstract}
We formulate and justify three \emph{Transdimensional Thermodynamics} (TDT) laws that govern snap events on the ledger \( \LL \): (TDT–1) the capacity–payload balance \( \Delta \SL=\Delta \Sinfo \); (TDT–2) the area calibration that fixes the one-bit cost \( \Delta\Acal_{\text{one bit}}=4\ln 2\,\lp^{2} \) in 4D (\( \Delta\Acal_{D}=\APD/\T \) in \(D\) dimensions); and (TDT–3) a local, covariant statement that ledger entropy grows only at snaps via a delta-supported source proportional to the exterior entropy influx. These laws are consistent with the black-hole first law and the Clausius relation in local Rindler patches \cite{Wald1984,Poisson2004,Jacobson1995,Wald1993,Bardeen1973}, and fit hand-in-glove with the geometric mechanism (NPR) and lossless inner boundary (real–Robin) of DCT/QG parts~I-II \cite{DCTQG01,DCTQG02,Ishibashi2003,ReedSimon1975,Evans2010}. 
\end{abstract}

\tableofcontents

\section{Introduction and context}
DCT/QG part~I fixed the \emph{placement} of the ledger \( \LL \) \cite{DCTQG01} by the invariant curvature trigger
\[
\Ical = \Icrit = \frac{12}{\T}, 
\]
which yields \( r_{\LL}=\sqrt{\T}\,\RS \) and \( \AL/\AH=\T \) for the Schwarzschild family \cite{Wald1984,Poisson2004,DCTQG01}. Part~II defined the geometric mechanism of a snap, \emph{Null-Pair Removal} (NPR)\cite{DCTQG02}, and justified the \emph{real–Robin} inner boundary that makes the exterior evolution unitary and lossless between snaps \cite{DCTQG02,Ishibashi2003,ReedSimon1975,Evans2010}.

In this paper we add the missing thermodynamic layer: statements that tie the \emph{payload information} written at a snap to \emph{ledger capacity} and a \emph{geometric area tick}. The three boxed laws below are the main deliverables. 

\section{Setup and notation}

We use units \( c=\hbar=k_{B}=1 \), keeping the Newton constant \( G \) (or \( G_{D} \)) explicit. 
We reserve \( G_{D} \) exclusively for the \(D\)-dimensional gravitational coupling in the Einstein–Hilbert action. 
For clarity we denote the \(D\)-dimensional Planck “area” unit by \( {}^{(D)}\!A_{\mathrm P} \equiv {}^{(D)}\ell_{\mathrm P}^{\ D-2} \); 
in our normalization \( {}^{(D)}\!A_{\mathrm P} = G_{D} \).

Entropy is measured in \emph{nats}; one bit equals \( \ln 2 \) nats. 
The Ledger’s thermodynamic entropy (capacity) is
\[
S_{\mathcal L} \;=\; \frac{A_{\mathcal L}}{4\,\ell_{\mathrm P}^{2}} \;=\; \frac{A_{\mathcal L}}{4\,G} \quad\text{(4D)}
\qquad\text{and}\quad
{}^{(D)}S_{\mathcal L} \;=\; \frac{{}^{(D)}\!A_{\mathcal L}}{4\,{}^{(D)}\!A_{\mathrm P}} \;=\; \frac{{}^{(D)}\!A_{\mathcal L}}{4\,G_{D}} \quad(D\text{-dim}).
\]
One irreversible bit \(\big(\Delta \Sinfo=\ln 2\big)\) consumes
\[
\Delta \Acal_{\text{one bit}} \;=\; 4\ln 2\,\ell_{\mathrm P}^{2}\quad (4\text{D}), 
\qquad
\Delta \Acal_{D} \;=\; 4\ln 2\,{}^{(D)}\!A_{\mathrm P}\quad (D\text{-dim}).
\]

\paragraph{Information vs.\ thermodynamic entropy (definitions).}
We distinguish two a priori different entropies:
\begin{itemize}
  \item \emph{Information (Shannon/von Neumann) entropy} \(\Sinfo\): a measure of missing information about a variable or quantum state (units: nats unless noted; one bit \(=\ln 2\) nats).
  \item \emph{Thermodynamic (Boltzmann/Gibbs) entropy} \(\Sth\): the state function entering Clausius \(\delta Q = T\, d\Sth\) and the Bekenstein--Hawking relation \(S_{\mathrm{BH}} = A/(4\AP)\).
\end{itemize}
Unless stated otherwise, \(\Sth\) refers to black-hole or Ledger thermodynamic entropy derived from geometry. \(\Sinfo\) refers to the \emph{payload} written during a snap.

\newpage

\section{The three TDT laws}

\paragraph{Information--Thermodynamics Distinction}
\label{tdt:info-vs-thermo}
\emph{Conceptual non-identity.} Information entropy \(\Sinfo\) and thermodynamic entropy \(\Sth\) are distinct notions. In particular,
\[
\Sinfo \neq \Sth \ \text{in general},\qquad
\delta Q = T\,d\Sth \ \text{(Clausius)}
\]
and \(S_{\mathrm{BH}}=A/(4\AP)\) concerns \(\Sth\). 

\begin{tdt}[Capacity–payload balance]
At a snap, DCT imposes the \emph{capacity--payload constraint}
\[
\boxed{\quad \Delta \SL \;=\; \Delta \Sinfo \quad}
\]
%so that recording one payload bit (\(\Delta \Sinfo=\ln 2\)) \emph{requires} a Ledger area tick \(\Delta A_{\mathcal L}=4\ln 2\,\AP\) via \(\Sth|_\LL=A_{\mathcal L}/(4\AP)\).

\noindent\textit{Interpretation.} Writing \( \Delta \Sinfo \) nats of \emph{payload} at a snap increases ledger capacity by \( \Delta \SL \); reversible metadata \( (X,Y,Z) \)\cite{DCTQG} carry no capacity cost. \\
This is a \emph{matching rule} between distinct quantities, not a claim that \(\Sinfo\) and \(\SL\) are the same concept. 
\end{tdt}

\begin{tdt}[Area calibration]
\[
\boxed{
\quad \Delta \Acal_{\text{one bit}} = 4\,\ln 2\;\lp^{2} \quad (4\text{D}),
\qquad
\quad \Delta \Acal_{D}=\frac{\APD}{\T} \quad (D\text{-dim}).
}
\]

for a single irreversible payload bit. Equivalently, \( \Delta \SL=\ln 2 \) per payload bit. \\
\noindent\emph{Units:} one bit \(=\ln 2\) nats; hence \(\Delta \Acal=4\ln 2\,\AP\) matches  \( \SL=\ln 2\) via \( \SL=A_{\mathcal L}/(4\AP)\).

\end{tdt}

\begin{tdt}[Episodic growth — local covariant form]
\label{tdt:3}
Let \(s^{a}\) be a \emph{surface} entropy current on the Ledger worldtube \(\mathcal{W}_{\LL}\) (induced metric \(\gamma_{ab}\), volume element \(\mathrm dV_{\mathcal W}\)), and let \(\sigma_{\rm in}\) denote the exterior entropy influx density. Then, in the sense of distributions on \(\mathcal{W}_{\LL}\),
\[
\boxed{\quad \nabla_{a}s^{a} \;=\; \sum_{i} J_{(i)}\,\delta_{\Sigma_i} \quad}
\]
with \(J_{(i)}=\int_{\Sigma_i}\sigma_{\rm in}\,\mathrm d\Sigma\). Equivalently, for any slab \(\mathcal V\subset\mathcal{W}_{\LL}\) crossing \(N\) snaps,
\[
\int_{\mathcal V}\!\nabla_a s^a\,\mathrm dV_{\mathcal W} \;=\; \sum_{i=1}^{N}\! \Delta S_{\mathcal L}^{(i)} \;=\; \sum_{i=1}^{N}\!\int_{\Sigma_i}\!\sigma_{\rm in}\,\mathrm d\Sigma,
\]
and between snaps \( \nabla_a s^a=0\). Along an inward generator with affine parameter \(\lambda\),
\[
\frac{\mathrm d \SL}{\mathrm d \lambda}\Big|_{\text{snap}} \;=\; \sigma_{\rm in}\Big|_{\LL} \ge 0,
\qquad 
\frac{\mathrm d \SL}{\mathrm d \lambda}=0 \ \text{otherwise}.
\]
\end{tdt}

\begin{figure}[t]
  \centering
  \includegraphics[width=0.88\linewidth]{common/images/p03/tdt_ledger_evolution.pdf}
  \caption{%
    \textbf{TDT–3 (episodic growth) as a staircase.}
    Quiet phases have $\nabla_a s^a = 0$ (flat segments).
    At snaps, the Ledger writes payload $\Delta S_{\mathrm{info}}=\ln 2$ and increases capacity
    $\Delta \SL=\ln 2$, producing a jump
    $\Delta \Acal_{\mathcal L}=4\ln 2\AP$ via $\Sth|_\LL=A_{\mathcal L}/4\AP$; see TDT-\ref{tdt:3}.
  }
  \label{fig:tdt3-staircase}
\end{figure}

\newpage

\section{Derivations and justification}
\subsection*{Justification for TDT-1}
TDT-1 is the foundational axiom of the framework's information bookkeeping. It states a direct, one-to-one equivalence between the information-theoretic content of an irreversible payload (\(\Delta \Sinfo\) in nats) and the resulting change in the Ledger's geometric entropy capacity (\(\Delta\SL\)). Hence
\[
\Delta \SL = \Delta \Sinfo.
\]
This is motivated by the physical role of the Ledger:
\begin{itemize}[noitemsep]
    \item A snap changes the usable capacity of \(\LL\) only through the irreversible payload bit \(\W\).
    \item The reversible geometric metadata bits \((X,Y,Z)\) set the local frame and inner phase but, being reversible, can be written and un-written without thermodynamic cost and thus do not consume net capacity\cite{DCTQG}.
\end{itemize}
Therefore, the entire capacity increase must be equal to the payload written. The information-theoretic origin of the constant \(\T\) derived in part~I of the DCT/QG framework\cite{DCTQG01}, based on the minimal 1-in-4 bit structure of a reversible snap.

\subsection*{From BH entropy to TDT–2}
Using \( \SL=\T\,\SBH \) with \( \SBH=\AH/(4\lp^2) \) \cite{Bardeen1973,Wald1993,Wald1984}, a single payload bit gives
\[
\Delta \SL=\ln 2 
\quad\Rightarrow\quad
\Delta\Acal = 4\lp^2\,\Delta \SL
= 4\ln 2\,\lp^2 \quad (4D).
\]
In \( D \) dimensions, adopt \( \lPD^{\,D-2} \equiv \APD \), the area tick is equivalent to
\[
\Delta\Acal_{D}=\frac{\APD}{\T}.
\]
This shows that the area quantum in TDT–2 is perfectly consistent with the ledger-horizon relations and TDT–1. A more formal, first-principles derivation of this result from the theory's foundational information axiom is provided in Appendix \ref{app:areaquantum}.

\subsection*{Local covariant form TDT–3}
Define a surface current \( \entropycurrent^{a} \) on the worldtube of \( \LL \) such that \( \int_{\mathcal C} \entropycurrent^{a}\,\dd \Sigma_{a} = \Delta \SL \) for any cross-section \( \mathcal C \). During lossless evolution with a real–Robin boundary, no energy/entropy crosses \( \LL \) \cite{Ishibashi2003,Evans2010,ReedSimon1975}, hence \( \nabla_{a}\entropycurrent^{a}=0 \). At a snap, couple \( \entropycurrent^{a} \) to the coarse-grained entropy influx \( \entropyinflux \) associated with the predicate that defines \( \W \) (NPR)\cite{DCTQG02}; the only allowed source is a delta-supported layer whose strength is fixed by the incoming payload entropy flux. This yields the boxed law in TDT–3[\ref{tdt:3}].

\section{Consistency with GR thermodynamics}

\subsection*{Horizon response convention at a snap}
Two consistent bookkeeping choices exist:

\paragraph{Option H (horizon-locked).}
Impose \(A_{\mathcal L}/A_H=\T\) at all times. Then a Ledger tick
\(\Delta A_{\mathcal L}=4\ln 2\,\ell_P^2\) implies a concurrent horizon tick
\(\Delta \Acal_H=\Delta \Acal_{\mathcal L}/\T\), so
\[
\delta S_{\mathrm{BH}}=\frac{\Delta \Acal_H}{4\ell_P^2}=\frac{\ln 2}{\T},
\qquad
\delta M=T_H\,\delta S_{\mathrm{BH}}=\frac{T_H\,\ln 2}{\T}.
\]

\paragraph{Option L (ledger-local).}
Treat the snap as an interior write with no instantaneous horizon change:
\(\Delta \Acal_{\mathcal L}=4\ln 2\,\ell_P^2\), \(\Delta \Acal_H=0\) at the event. The ratio \(A_{\mathcal L}/A_H\) then relaxes back to \(\T\) under subsequent GR evolution; no horizon heat crosses at the instant.

\paragraph{Adopted convention.}
For algebraic continuity with ledger placement in part~I of DCT/QG\cite{DCTQG01} we adopt \textbf{Option H} in what follows; \textbf{Option L} leads to the same coarse-grained predictions after relaxation.

\subsection*{First law check (Kerr–Newman form)}
The first law reads \cite{Bardeen1973,Wald1993,Wald1984,Poisson2004}
\[
\dd M \;=\; T_H\,\dd \SBH + \Omega_H\,\dd J + \Phi_H\,\dd Q,
\qquad \SBH=\frac{\AH}{4\lp^2}.
\]
We check consistency at fixed \( (J,Q) \) (\(\dd J=\dd Q=0\)). At a \emph{snap} (Option H), one payload bit gives
\[
\Delta \Sinfo=\ln 2,
\qquad
\Delta \SL=\ln 2
\quad\text{(TDT--1).}
\]
Using the placement relation \( \SL=\T\,\SBH \), the horizon change is
\[
\Delta \SBH \;=\; \frac{\Delta \SL}{\T} \;=\; \frac{\ln 2}{\T}.
\]
Interpreting the first law across this finite step as the energetic cost of writing this one bit to the black hole is therefore
\[
\Delta M \;=\; T_H\!\big|_{\text{snap}}\; \Delta S_{\mathrm{BH}} \;=\; T_H\!\big|_{\text{snap}}\,\frac{\ln 2}{\mathcal T}.
\]
This matches the area quantum (TDT--2): \( \Delta \AL=4\ln 2\,\lp^2 \), hence
\[
\Delta \AH \;=\; \frac{\Delta \AL}{\T},
\qquad
\Delta \SBH \;=\; \frac{\Delta \AH}{4\lp^2} \;=\; \frac{\ln 2}{\T}.
\]
All relations are internally consistent.

\subsection*{Clausius sketch (local Rindler)}
In Jacobson’s argument, imposing \( \delta Q = T\,\delta S \) across local Rindler horizons with \( \delta S \propto \delta A \) yields Einstein’s equations \cite{Jacobson1995}. On \( \LL \) the entropy functional satisfies \( S_{\LL} = A_{\LL}/(4\lp^2) \), so formally
\[
\mathrm d S_{\LL} \;=\; \frac{1}{4\lp^2}\,\mathrm d A_{\LL}.
\]
However, by the lossless boundary (real--Robin) condition, between snaps
\[
\delta Q\big|_{\LL}=0 \;\Rightarrow\; \mathrm d S_{\LL}=0, \qquad \nabla_a s^a=0.
\]
At snaps the change is \emph{discrete} and must be written with uppercase jumps:
\[
\Delta S_{\LL}=\ln 2,
\qquad
\Delta A_{\LL}=4\ln 2\,\lp^2.
\]
When comparing Ledger and horizon responses, the placement ratio \( \frac{A_{\LL}}{A_H}=\T \) implies
\(
\Delta \SBH=\Delta \SL/\T
\),
reproducing the factor \(1/\T\) seen in the first-law check.

\section{Dynamics interface: NPR, Robin, and the impulse}
NPR part of DCT shows snaps are instantaneous maps comprising (i) state dephasing \( \rho\mapsto \sum_r \Pi_r \rho \Pi_r \) (writing \( \W \)) \cite{NielsenChuang2010,Breuer2002}, (ii) NPR deletion of normal components, and (iii) a focused Raychaudhuri impulse that ticks area by \( \Delta\Acal \), ensuring finite focusing:
\[
\int_{\lambda^-}^{\lambda^+} \theta\,\dd \lambda \;=\; \ln\!\Big(1+\frac{\Delta\Acal}{A}\Big) \ \approx\ \frac{\Delta\Acal}{A}.
\]
Between snaps, real–Robin implies unit-modulus reflectivity and self-adjoint exterior dynamics \cite{Ishibashi2003,ReedSimon1975,Evans2010}.

\section{D-dimensional generalization}
With \( \lPD^{\,D-2}=\APD \), the capacity and tick generalize to
\[
\SL=\frac{\lDtag \AL}{4\,\APD},
\qquad
\Delta\Acal_{D}=\frac{\APD}{\T}.
\]
From part~I, the placement/area ratios along the Tangherlini ladder satisfy
\[ 
\frac{\lDtag r_{\LL}}{\lDtag \rH}=\T^{\frac{1}{2(D-3)}}, 
\qquad 
\frac{\lDtag \AL}{\lDtag \AH}=\T^{\frac{D-2}{2(D-3)}}, 
\]
so the \emph{entropy} cost per snap is \( D \)-independent (one bit \(\Rightarrow \ln 2\) nats of payload), while areas/radii scale with \( D \) as above \cite{Poisson2004,DCTQG01}.

\section{Capacity, used space, and remnants}
Define the total capacity (in bits) and the cumulative count of irreversible payload writes:
\[
C(M)\equiv \frac{\AL}{4\ln 2\,\lp^{2}}, \qquad N_{\mathrm{irr}}(t)\ \text{non-decreasing (snaps only)}.
\]
Then
\[
C\big(M(t)\big)\ \ge\ N_{\mathrm{irr}}(t),
\]
with equality at a stabilized remnant where evaporation has reduced \( C \) to the used payload \( N_{\mathrm{irr}} \). In 4D, this gives a parametric floor
\[
M_{\mathrm{rem}} \;\sim\; \frac{\ln 2}{\sqrt{\pi}}\,\sqrt{N_{\mathrm{irr}}}\;M_{\mathrm P},
\]
modulo \emph{kinematic} floors from mode structure\cite{DCTQG}. 

\section{Worked mini-examples}
\paragraph{One-bit snap (4D).}
Given \( \Delta \Sinfo=\ln 2 \), TDT–1 yields \( \Delta \SL=\ln 2 \) and TDT–2 gives \( \Delta\Acal=4\ln 2\,\lp^{2} \). 

\paragraph{Two separated snaps.}
Between snaps, \( \nabla_a \entropycurrent^{a}=0 \) (lossless Robin), so the total jump in \( \SL \) is the sum of the two impulses, each proportional to the respective \( \entropyinflux \) at that event.

\paragraph{\(D\)-dimensional tick.}
With \( \GD \) fixed, a single payload bit costs \( \Delta\Acal_{D}=\APD/\T \) irrespective of \( D \); geometric scaling enters only through the placement/area ratios.

\newpage

\section{Discussion and outlook}
TDT fixes the ledger’s thermodynamic bookkeeping: how much capacity must be provisioned when a payload bit is written, how that translates into an area tick, and why ledger entropy grows only episodically. It \emph{does not} decide \emph{when} a snap occurs—that depends on the curvature trigger, NPR, and decoherence predicates \cite{DCTQG01, DCTQG02}. Upcoming in the DCT/QG series: we will formalize the reversible \( (X,Y,Z) \) bits; derive the focused Raychaudhuri impulse; develop GW echo phenomenology with inner phases set by this metadata.

\appendix
\section{Formal Derivation of the Area Quantum}
\label{app:areaquantum}

This appendix provides the rigorous, first-principles derivation of second law of TDT, the universal area quantum. This derivation serves as the formal proof of Postulate P1 in part~I of the DCT/QG framework, which was used in the ledger placement and derivation of \(\T\) constant. The argument proceeds from the theory's most fundamental axiom regarding the information structure of a snap.

\begin{enumerate}[leftmargin=1.5em]
    \item \textbf{The Foundational Information Axiom:} The theory's starting point, justified heuristically in Section 4.1, is that a snap event has a minimal four-bit structure with only one irreversible payload bit (\(\W\)). This fixes the payload fraction to \(p=1/4\) and thereby determines the universal constant \(\T = 1/(4\ln 2)\) through the full QES analysis in part~I.

    \item \textbf{Entropy Cost per Bit (TDT–1):} The first law of TDT, states that the ledger capacity increase (in nats) equals the payload information written. For a single payload bit, this cost is:
    \[
    \Delta \SL = \Delta \Sinfo = \ln 2.
    \]

    \item \textbf{Deriving the Area Cost:} The ledger's entropy is defined by the standard geometric formula, \(\SL = \AL/(4\lp^2)\). A change in the ledger's entropy must therefore correspond to a change in its area according to \(\Delta \AL = 4\lp^2\,\Delta \SL\). By substituting the entropy cost for one bit from the previous step, we derive the precise area quantum:
    \[
    \Delta\Acal_{\text{one bit}} = 4\lp^2 \times (\ln 2) = 4\ln 2\,\lp^{2}.
    \]
\end{enumerate}
This result is TDT–2 in four dimensions. This derivation demonstrates that the specific area cost per bit is not an independent postulate but is a direct and necessary consequence of TDT–1 and the fundamental information-theoretic axiom that fixes \(\T\). This completes the proof of Postulate P1\cite{DCTQG01}.

\bibliographystyle{unsrt}
\bibliography{bibliography/dct_refs, bibliography/external_refs}

\end{document}












