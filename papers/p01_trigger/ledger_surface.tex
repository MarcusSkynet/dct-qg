\documentclass[11pt,a4paper]{article}

% =======================================================================
% == CANONICAL UNIFIED PREAMBLE FOR THE DCT/QG SERIES (v2.0) ==
% =======================================================================

% ------------------- Document Class & Core Setup -------------------
% Recommended base class.
% \documentclass[11pt,a4paper]{article}

% ------------------- Essential Packages -------------------
% --- Font, Encoding, and Microtypography ---
\usepackage[T1]{fontenc}
\usepackage[utf8]{inputenc}
\usepackage{lmodern}      % For crisp, scalable fonts
\usepackage{microtype}    % Improves the typographic quality of the document

% --- Page Layout & Spacing ---
\usepackage{geometry}
\geometry{margin=1in} % Standard 1-inch margins
\usepackage{setspace}
\onehalfspacing        % 1.5 line spacing for readability

% --- Mathematics & Physics Packages ---
\usepackage{amsmath,amssymb,mathtools,amsthm}
\usepackage{bm}           % For bold math symbols (\bm)
\usepackage{upgreek}      % For upright Greek letters (e.g., \upmu)
\usepackage{mathrsfs}     % For script font (\mathscr)
\usepackage{siunitx}      % For professional unit typesetting (\SI, \si)
\usepackage{physics}      % For commands like \dd, \Tr, \ket, \bra, etc. (replaces some manual defs)

% --- Citations, Links, Graphics, and Notes ---
\usepackage{hyperref}
\hypersetup{colorlinks=true,linkcolor=blue,citecolor=blue,urlcolor=blue}
\usepackage{graphicx}
\usepackage{enumitem}     % For customized lists
\usepackage{orcidlink}    % For ORCID iDs
\usepackage{tcolorbox}    % For styled boxes
\usepackage{verbatim}     % For code blocks
\usepackage{listings}     % For code listings
\usepackage{csquotes}     % For context-sensitive quotation marks (\enquote)
\usepackage{todonotes}    % For To-Do notes in the margin

% --- Optional but Recommended Packages ---
\usepackage{tikz}         % For drawing figures programmaticallly

% ------------------- Theorem-like Environments -------------------
% Standardized environments for consistent labeling.
\theoremstyle{plain}
\newtheorem{theorem}{Theorem}
\newtheorem{proposition}{Proposition}
\newtheorem{lemma}{Lemma}
\theoremstyle{definition}
\newtheorem{definition}{Definition}
\newtheorem{law}{Law}
\newtheorem{tdt}{TDT} % Specific environment for TDT laws if needed
\theoremstyle{remark}
\newtheorem{remark}{Remark}

% --- Custom Boxed Environment for Scope Remarks ---
\newtcolorbox{scopebox}[1][]{
  colback=gray!5,
  colframe=gray!75!black,
  fonttitle=\bfseries,
  title={#1},
  % If no title is given by the user, default to "Scope & Open Items"
  IfValueTF={#1}{}{title={Scope \& Open Items}}
}

% =======================================================================
% == CANONICAL MACROS (Grouped by Function) ==
% =======================================================================

% ------------------- 1. DCT Foundational Concepts (Calligraphic) -------------------
\newcommand{\LL}{\mathcal{L}}          % The Ledger surface
\newcommand{\Ical}{\mathcal{I}}        % The universal curvature invariant
\newcommand{\Icrit}{\Ical_{\mathrm{crit}}} % The critical value of the invariant
\newcommand{\T}{\mathcal{T}}           % The Transdimensional Constant
\newcommand{\Jcal}{\mathcal{J}}        % The snap instrument superoperator
\newcommand{\Rcal}{\mathcal{R}}        % The local areal radius (the geometric "ruler")
\newcommand{\Acal}{\mathcal{A}}        % Dynamical/infinitesimal area (e.g., in Raychaudhuri eq. and for area ticks)

% ------------------- 2. Boundary Dynamics (Sans-Serif) -------------------
\newcommand{\Bop}{\mathsf{B}}          % The Robin boundary operator
\newcommand{\Rrefl}{\mathsf{R}}        % The reflection coefficient
\newcommand{\SBC}{S_{\text{BC}}}       % ACTION of the Boundary Condition sector
\newcommand{\Kboundary}{\mathsf{K}}    % General boundary kernel operator (contextual)

% ------------------- 3. Actions & Lagrangians -------------------
\newcommand{\Lag}{\mathscr{L}}         % Lagrangian density (script font)
\newcommand{\SL}{S_{\mathcal{L}}}      % *** ENTROPY of the Ledger ***
\newcommand{\Sledger}{S_{\text{ledger}}} % *** ACTION of the Ledger EFT ***

% ------------------- 4. Standard Physics & GR Quantities -------------------
% --- General ---
\newcommand{\lp}{\ell_{\mathrm P}}     % Planck length
\newcommand{\MP}{M_{\mathrm P}}        % Planck mass
\newcommand{\br}{\mathrm {br}}         % The irreversible payload bit
% --- Black Holes ---
\newcommand{\RS}{R_{\mathrm S}}        % Schwarzschild radius
\newcommand{\EH}{\mathrm{H}}           % Event Horizon label (e.g., A_H, S_H)
\newcommand{\rH}{r_{\mathrm H}}        % General D-dimensional horizon radius
\newcommand{\rL}{r_\LL}                % Ledger radius
\newcommand{\AL}{A(\LL)}               % Area of the ledger
\newcommand{\AH}{A_{\mathrm H}}        % Area of the horizon
\newcommand{\THaw}{T_{\mathrm H}}      % Hawking temperature
% --- GR & Geometry ---
\newcommand{\Kretsch}{K}               % Kretschmann scalar
\newcommand{\RRm}{R_{ABCD}R^{ABCD}}    % Kretschmann scalar written out
\newcommand{\thet}{\theta}             % Expansion of a null congruence
\newcommand{\sig}{\sigma_{\mu\nu}\sigma^{\mu\nu}} % Shear norm
\newcommand{\Ric}{R_{\mu\nu}k^\mu k^\nu} % Ricci focusing term for null geodesics

% ------------------- 5. Mathematical Helpers -------------------
\newcommand{\RR}{\mathbb{R}}          % Real numbers
\newcommand{\Id}{\mathbb{I}}           % Identity operator/matrix
\newcommand{\SO}{\mathrm{SO}}         % Special Orthogonal group
\newcommand{\nplus}{n_{+}}             % Future-directed outgoing null normal
\newcommand{\nminus}{n_{-}}            % Future-directed ingoing null normal
\newcommand{\Pproj}{P^{A}{}_{B}}       % The NPR projector
\newcommand{\sgn}{\operatorname{sgn}}  % Sign function

% ------------------- 6. Quantum Mechanics Helpers -------------------
\newcommand{\Hhat}{\widehat{H}}        % Hamiltonian operator
\newcommand{\Uhat}{\widehat{U}}        % Unitary evolution operator

% ------------------- 7. Dimensional Tagging Helper -------------------
% Usage: \dimtag{D}{R} produces the D-dimensional Ricci scalar.
\newcommand{\dimtag}[2]{{}^{(#1)}\!{#2}}
\newcommand{\Dtag}{(D)}                % Dimensional tag
\newcommand{\dtag}{(d)}                % D-2 Dimensional tag
\newcommand{\lDtag}{{}^{\Dtag}}        % Left dimensional tag
\newcommand{\ldtag}{{}^{\dtag}}        % Left D-2 dimensional tag

% ------------------- 8. Numerical Anchors -------------------
\newcommand{\numTval}{0.36067376}      % 1/(4 ln 2)
\newcommand{\numIcritval}{33.27106467} % 48 ln 2

% ------------------- 9. Phenomenological Modules -------------------
% --- 9a. Echoes & Scattering ---
\newcommand{\El}{\mathsf{E}_{\ell}}    % Echo transfer function
\newcommand{\tauRT}{\tau}              % Round-trip time / Echo delay parameter
\newcommand{\phase}{\varphi}           % The phase angle symbol
\newcommand{\Rreflw}{\Rrefl_{\mathrm w,\ell}} % Wall reflectivity
\newcommand{\Rreflb}{\Rrefl_{b,\ell}}  % Barrier reflectivity
\newcommand{\Tw}{\mathsf{T}_{\mathrm w,\ell}} % Wall transmissivity
\newcommand{\Tb}{\mathsf{T}_{b,\ell}}  % Barrier transmissivity
% --- 9b. Tensor Networks & QEC ---
\newcommand{\Viso}{V}                  % Node isometry
\newcommand{\Xtot}{X_{\mathrm{tot}}}   % Total X operator
\newcommand{\Ytot}{Y_{\mathrm{tot}}}   % Total Y operator
\newcommand{\Ztot}{Z_{\mathrm{tot}}}   % Total Z operator
% --- 9c. Dark Matter & Remnants ---
\newcommand{\wgap}{\omega_{\mathrm{gap}}} % Evaporation mass gap frequency
\newcommand{\Mrem}{M_{\mathrm{rem}}}   % Remnant mass
\newcommand{\gH}{\gamma_{\mathrm H}}      % Hawking effective greybody coefficient
\newcommand{\chiH}{\chi}                  % Robin suppression factor
\newcommand{\betaf}{\beta_{\!f}}          % Initial PBH fraction at formation
% --- 9d. Dark Energy & Cosmology ---
\newcommand{\Hcal}{\mathcal{H}}         % Conformal Hubble parameter
\newcommand{\Cdot}{\dot{\mathcal{C}}}   % Ledger capacity growth rate
\newcommand{\rhoL}{\rho_{\LL}}          % Ledger energy density
\newcommand{\pL}{p_{\LL}}               % Ledger pressure
\newcommand{\wL}{w_{\LL}}               % Ledger equation of state
\newcommand{\csL}{c_{s,\LL}^2}        % Ledger sound speed
\newcommand{\OM}{\Omega_{\mathrm m}}    % Matter density parameter
\newcommand{\OL}{\Omega_{\LL}}          % Ledger (DE) density parameter
\newcommand{\ORad}{\Omega_{\mathrm r}}    % Radiation density parameter
% --- 9e. Radion & Fifth Force ---
\newcommand{\MD}{M_{D}}                % D-dimensional Planck mass
\newcommand{\varphiRad}{\varphi}       % Canonical radion field
\newcommand{\mn}{m_n}                   % KK mode mass
\newcommand{\Lef}{L_{\mathrm{eff}}}    % Effective compactification length
\newcommand{\lambdaff}{\lambda_{\mathrm{5th}}} % Fifth-force range

% =======================================================================
% == DEPRECATED / DUPLICATED MACROS (For Search & Replace) ==
% =======================================================================
% These are kept here (commented out) to help you find and replace
% old commands in your documents.

% %\newcommand{\aH}{\alpha_{\mathrm H}}     % DEPRECATED. Use \T instead.
% %\newcommand{\CurvTrigger}{\Ical=\Icrit} % DEPRECATED. Use the equation directly.
% %\newcommand{\Area}{\mathcal A}          % DEPRECATED. Use \Acal.
% %\newcommand{\Bboundary}{\mathsf{B}}      % DUPLICATE of \Bop.
% %\newcommand{\Rw}{\Rrefl_{\mathrm w,\ell}}% DUPLICATE of \Rreflw.
% %\newcommand{\Rb}{\Rrefl_{b,\ell}}      % DUPLICATE of \Rreflb.
% %\newcommand{\MP}{M_{\mathrm P}}           % DUPLICATE of M_P macro.
% %\newcommand{\Mpl}{M_{\mathrm{Pl}}}      % DUPLICATE of M_P macro.
% %\newcommand{\PNPR}{\mathsf P_{\mathrm{NPR}}} % DEPRECATED. Use \Pproj.
% %\newcommand{\tRT}{\tau_{\mathrm{RT}}}   % DEPRECATED. Use \tauRT.
% %\newcommand{\omegaGap}{\omega_{\mathrm{gap}}} % DUPLICATE of \wgap.
% %\newcommand{\Deltaecho}{\Delta t_{\mathrm{echo}}} % DEPRECATED. Use \tauRT.
% %\newcommand{\lPD}{{}^{(D)}\ell_{\mathrm P}} % DEPRECATED. Use \dimtag{D}{\lp}.
% %\newcommand{\Dtag}{(D)}                 % DEPRECATED. Use \dimtag.
% %\newcommand{\dtag}{(d)}                 % DEPRECATED. Use \dimtag.
% %\newcommand{\lDtag}{{}^{\Dtag}}         % DEPRECATED. Use \dimtag.
% %\newcommand{\ldtag}{{}^{\dtag}}         % DEPRECATED. Use \dimtag.



% ====== Title ======
\title{
    \bfseries The Universal Interior Surface of Black Holes \\[4pt] 
    \large and the Derivation of the Transdimensional Constant
}

\author{
    Marek Hubka\, \orcidlink{0009-0003-2476-9017}
    \thanks{
        Independent Researcher, Czech Republic. \\
        Website: \href{http://www.tidesofuncertainty.com}{tidesofuncertainty.com}.
        Email: \href{mailto:marek@tidesofuncertainty.com}{marek@tidesofuncertainty.com}.
    }
}
\date{October 7, 2025}
% \date{\today}

\begin{document}
\maketitle
\vspace{-2ex}

\begin{abstract}
We derive, from first principles, the location of a universal interior surface \(\LL\) inside stationary black holes where the generalized expansion vanishes. In four dimensions we obtain a universal, dimensionless constant \(\T=1/(4\ln 2)\) (the \emph{Transdimensional Constant}), which fixes the local curvature threshold
\(\Icrit=48\ln 2\) for the dimensionless invariant \(\Ical(r)\equiv K(r)\,r^{4}\) (with \(K\) the Kretschmann scalar) and places the ledger at \(r_{\LL}=\sqrt{\T}\,R_S\). Equivalently, in 4D the ledger’s area fraction equals \(\T\): \(\AL/\AH=\T\).
Our reasoning has two legs. \emph{Orthodox leg (GR + semiclassical QFT)}: a lossless, self-adjoint inner boundary implies a QES condition \(\Theta_{\text{gen}}=0\) whose solution selects \(\Ical(r)=\Icrit=24/(4\lp^{2}\gamma_{\mathrm{rob}})\), with a single universal one-loop number \(\gamma_{\mathrm{rob}}\) from the replica/heat-kernel coefficient on a cone with real-Robin boundary \cite{Vassilevich2003, Fursaev1995}. \emph{Closure leg (minimal information principles)}: black-hole thermodynamics in bits, together with the minimal reversible record at snaps, fixes \(\gamma_{\mathrm{rob}}=\ln 2/(2\lp^{2})\), hence \(\T=1/(4\ln 2)\) with no adjustable parameters. We also check energy accounting \(E_{\text{slab}}=M\), quantum focusing, and higher-dimensional continuation.
\end{abstract}

\tableofcontents

\section{Setup and invariant ruler}

Consider a four-dimensional, stationary, asymptotically flat black hole (Schwarzschild for concreteness) of mass \(M\), with horizon radius \(R_{S}=2M\). The Kretschmann scalar is
\[
K(r)=\RRm=\frac{48M^{2}}{r^{6}}.
\]
Define the \emph{local, dimensionless} curvature ruler
\[
\Ical(r)\;\equiv\;K(r)\,r^{4}.
\]
Label any round interior 2-sphere by the dimensionless area fraction
\[
\alpha \;\equiv\;\frac{A(r)}{\AH}=\Big(\frac{r}{R_{S}}\Big)^{2}
\quad\Longrightarrow\quad
\Ical(R_{\mathrm S})=12,\qquad
\Ical(r)=\frac{12}{\alpha}.
\]
The aim is to determine, from first principles, the \emph{constant} \(\T\) and the physically admissible interior surface \(\LL\); in 4D this will imply \(\AL/\AH=\T\) and \(r_\LL=\sqrt{\T}\,R_S\)

\section{Orthodox leg I: lossless inner boundary from unitarity}

Excise the deep interior at a trial sphere \(r\). Requiring the exterior dynamics to be unitary imposes that the radial Hamiltonian on a half-line be self-adjoint. For each partial-wave channel this enforces a \emph{real--Robin} boundary condition at \(r\),
\[
\big(\partial_{r_*}+\mathsf B\big)\,\psi\big|_{r}=0,\qquad \mathsf B\in\RR,
\]
where \(r_*\) is the tortoise coordinate. The associated reflection amplitude obeys
\[
\mathsf R(\omega)=\frac{\mathsf B-i\omega}{\mathsf B+i\omega}\,e^{2i\varphi(\omega)},\qquad |\mathsf R(\omega)|=1,
\]
so the surface is \emph{lossless}: there is no normal energy flux through \(r\). This step fixes \emph{what kind} of inner surface is allowed (unit-modulus reflection), not \emph{where} it is located.


\section{Orthodox leg II: QES stationarity and the one-number reduction}

Let \(k^{a}\) be the inward null generator orthogonal to a round sphere at \(r\). The generalized expansion reads \cite{Wall2012}
\[
\Theta_{\text{gen}}(r) \;=\; \underbrace{\theta(r)}_{\text{GR}} \;+\; 4\lp^{2}\,\underbrace{\partial_{\lambda}\!\Big(\frac{S_{\mathrm{out}}}{\delta \Acal}\Big)(r)}_{\text{QFT on the same background and boundary}},
\]
and a quantum extremal surface (QES) satisfies \(\Theta_{\text{gen}}=0\).

\paragraph{Geometric piece.} For a round sphere pushed inward along \(k^{a}\),
\[
\theta(r) \;=\; -\frac{2}{r} + \mathcal{O}\Big(\frac{R_{S}}{r^{2}}\Big).
\]

\paragraph{QFT piece (structure).} In a local Rindler frame anchored on the sphere, with the \emph{same} real--Robin boundary, the null first-law integral and the replica/heat-kernel analysis on the cone localize the entropy response on the surface and yield the universal form
\[
\partial_{\lambda}\!\Big(\frac{S_{\mathrm{out}}}{\delta A}\Big)(r)
\;=\; \frac{\gamma_{\mathrm{rob}}}{r}\;\frac{\Ical(r)}{12},
\]
where \(\gamma_{\mathrm{rob}}\) is a \emph{single} dimensionless one-loop coefficient determined by the Seeley--DeWitt data on the cone with real--Robin boundary. The only \(r\)-dependence outside \(I\) is the common affine normalization \(1/r\).

\paragraph{Stationarity selects a constant \(\Icrit\).} Insert both pieces into \(\Theta_{\text{gen}}=0\); the factor \(1/r\) cancels:
\[
-\,2 \;+\; 4\lp^{2}\,\gamma_{\mathrm{rob}}\,\frac{\Ical(r)}{12}\;=\;0
\quad\Longrightarrow\quad
\Ical(r)=\Icrit\;\equiv\;\frac{24}{4\lp^{2}\,\gamma_{\mathrm{rob}}}.
\]
Using \(\Ical(r)=12/\alpha\),
\[
\boxed{\;\alpha\;=\;\frac{12}{\Icrit}\;=\;2\,\lp^{2}\,\gamma_{\mathrm{rob}}\; }.
\]
At this point, the location of \(\LL\) is reduced to \emph{one} universal number \(\gamma_{\mathrm{rob}}\).

\paragraph{Where \(\gamma_{\mathrm{rob}}\) comes from (orthodox content).}
In four dimensions, with entropy measured in \emph{bits}, the replica/heat-kernel dictionary gives
\[
\gamma_{\mathrm{rob}}=\frac{\tilde c_{\mathrm{rob}}}{4\pi\,\ln 2},
\qquad
\tilde c_{\mathrm{rob}} \equiv c_{\text{bulk}} + c_{\text{bdry}}^{(\text{Robin})},
\]
where \(c_{\text{bulk}}\) is the standard cone coefficient (e.g.\ \(1/90\) for a real scalar) and \(c_{\text{bdry}}^{(\text{Robin})}\) is the boundary contribution for a real--Robin condition on a smooth 2-sphere; both are tabulated Seeley--DeWitt data \cite{Vassilevich2003}. Summing the field content yields a definite \(\tilde c_{\mathrm{rob}}\) and thus \(\gamma_{\mathrm{rob}}\).

\paragraph{Orthodox hand-off.} The orthodox derivation thus fixes \(\alpha\) \emph{up to} the single, standard one-loop constant \(\gamma_{\mathrm{rob}}\). No additional physics beyond GR, QFT in curved spacetime, and BH thermodynamics has been used.

\newpage

\section{Closure leg: fixing \texorpdfstring{\(\gamma_{\mathrm{rob}}\)}{gamma} from minimal information principles}

We now close the last gap using two principles that are themselves part of the established thermodynamic structure and of a conservative microscopic bookkeeping at curvature-threshold events:

\begin{itemize}[leftmargin=1.35em]
\item \textbf{(P1) Bit-normalized BH density\cite{DCTQG}.} Expressed in \emph{bits}, the Bekenstein--Hawking law fixes a universal surface density % Cite DCTQG03 when available
\[
\frac{S}{A}\Big|_{\text{bits}} \;=\; \frac{1}{4\ln 2}\,\frac{1}{\lp^{2}},
\]
i.e.\ an irreversible \emph{payload} bit costs area \(\Delta \Acal=4\ln 2\,\lp^{2}\).

\item \textbf{(P2) Minimal reversible record at snaps\cite{DCTQG}.} At a curvature-threshold event (snap) the ledger records a four-bit register \(\mathrm{br},X,Y,Z\}\), of which \emph{only} the payload bit \(\mathrm{br}\) is thermodynamic (irreversible); the three geometric bits \(X,Y,Z\) encode reversible metadata that ensure lossless, self-adjoint evolution (no dissipation).
\end{itemize} % Cite DCTQG04 when available

At a \emph{lossless} cut there are two null directions sharing the local entanglement response symmetrically. Calibrated in bits via (P1), and with no dissipative leakage thanks to (P2), the null shape derivative per area is therefore fixed to
\[
\partial_{\lambda}\!\Big(\frac{S_{\mathrm{out}}}{\delta \Acal}\Big)
\;=\; \underbrace{\frac{\ln 2}{2\,\lp^{2}}}_{\displaystyle \gamma_{\mathrm{rob}}}\,\frac{1}{r}\,\frac{\Ical(r)}{12}.
\]
Thus \(\gamma_{\mathrm{rob}}=\ln 2/(2\lp^{2})\) without adjustable parameters. Plugging into the orthodox stationarity yields
\[
\Icrit \;=\; \frac{24}{4\lp^{2}\,\gamma_{\mathrm{rob}}}
\;=\; \frac{24}{4\lp^{2}\,(\ln 2/2\lp^{2})}
\;=\; 48\,\ln 2.
\]
Equivalently,
\[
\Icrit \;=\; 48\,\ln 2 \\
\]
\[\Downarrow\]
\[
\boxed{\;\T\;=\;\frac{1}{4\ln 2},\qquad
r_{\LL}=\sqrt{\T}\,R_{S}=\frac{R_{S}}{\sqrt{4\ln 2}},\qquad
\frac{\AL}{\AH}=\T\ \text{(in 4D)}\; }.
\]

\section{Consistency Checks and Foundational Results}

We now verify that the framework is self-consistent and leads to profound physical consequences.

\paragraph{Mass Independence and Locality.}
The ledger placement rule, \(\Ical(r_\LL) = \Icrit\), is constructed from a dimensionless scalar invariant. It is therefore independent of the black hole's mass and is defined by the local spacetime geometry, ensuring a universal and observer-independent criterion.

\paragraph{Energy Accounting and the Origin of Mass.}
With a lossless inner boundary at \(r_{\LL}\), the total quasi-local energy \(E_{\text{slab}}\) contained within the spacetime region between the ledger and the event horizon is given by the Misner-Sharp mass difference, \(E_{\text{slab}}(v) \equiv M(v) - m(r_{\LL})\). The change in this energy is governed by the flux across its boundaries. As no energy flux can cross the lossless ledger (\(\dd m(r_{\LL})/\dd v = 0\)), any change in the slab's energy must be equal to the energy flux crossing the event horizon:
\[
\frac{\dd E_{\text{slab}}}{\dd v} = \int_{R_{\mathrm S}} T_{ab}k^a k^b\,\dd\mathcal{A} = \frac{\dd M}{\dd v}.
\]
Integrating this relation gives \(E_{\text{slab}} = M + \text{const}\). By setting the physically necessary boundary condition that a zero-mass black hole must contain zero slab energy, the constant of integration vanishes. This leads to a profound result:
\[
\boxed{ E_{\text{slab}} = M. }
\]
This demonstrates that the black hole's entire mass-energy, as measured from infinity, is physically stored as the energy of the curved vacuum in the "slab" region between the ledger and the event horizon. The classical singularity is not needed to contain the mass; it is non-locally distributed in the gravitational field itself.

\paragraph{Quantum Focusing and Stability.}
The ledger's placement is consistent with the Quantum Focusing Conjecture (QFC), which is believed to be a fundamental law of quantum gravity. The QFC states that the expansion of a null congruence, when generalized to include entropy gradients (\(\Theta_{\text{gen}}\)), must be non-increasing (\(\dd\Theta_{\text{gen}}/\dd\lambda \le 0\)). We identify the ledger \(\LL\) as the stable surface where this generalized expansion is zero, \(\Theta_{\text{gen}}(r_{\LL})=0\). This makes the ledger a uniquely stable boundary: any hypothetical surface placed further out would violate the QFC, while any surface placed further in would represent a state of runaway focusing. The ledger is therefore the outermost possible stable quantum-gravitational boundary inside the event horizon.

\paragraph{Flat-Space Limit.}
As the mass of the black hole approaches zero (\(M\to 0\)), the event horizon radius \(R_{S}\) and the ledger radius \(r_{\LL}\) also shrink to zero. The slab region vanishes, and its energy \(E_{\text{slab}}=M\) correctly goes to zero. The construction smoothly reduces to empty, flat spacetime, satisfying a crucial consistency check.

\section{Higher-dimensional continuation}

For a Tangherlini black hole in \(D\) dimensions, the Kretschmann scalar is
\[
{}^{(D)}\!K(r)
=\frac{C_{D}\,r_{H}^{\,2(D-3)}}{r^{\,2(D-1)}},
\qquad
C_{D}=(D-1)(D-2)^{2}(D-3).
\]
Define the dimensionless invariant as in the manifest,
\[
{}^{(D)}\!\Ical(r)\;\equiv\;{}^{(D)}\!K(r)\,L^{4},
\qquad\text{with }\; L=r \ \text{for spherical symmetry}.
\]
Then at the horizon and at a general radius,
\[
{}^{(D)}\!\Ical(\mathrm{R_{S}})=C_{D},
\qquad
\frac{{}^{(D)}\!\Ical(r)}{{}^{(D)}\!\Ical(\mathrm{r_{H}})}
=\alpha^{-\frac{2(D-3)}{D-2}},
\qquad
\alpha \;\equiv\; \Big(\frac{r}{r_{H}}\Big)^{D-2}.
\]
Matching the same bit–density factor \(4\ln 2=1/\T\) at the Ledger gives
\[
\boxed{\
\alpha^{(D)}_{\LL}=(4\ln 2)^{-\frac{D-2}{2(D-3)}},\qquad
\frac{r_{\LL}}{r_{H}}=(4\ln 2)^{-\frac{1}{2(D-3)}},\qquad
{}^{(D)}\!\Icrit=C_{D}\,4\ln 2\ }.
\]
For \(D=4\) this reduces to \({}^{(4)}\!\Icrit=48\ln 2\) and \(r_{\LL}=\sqrt{\T}\,R_{S}\) with \(\T=1/(4\ln 2)\).


\section{Discussion and outlook}

We have split the derivation of \(\T\) into an orthodox leg and a closure leg. The orthodox leg, based entirely on GR, semiclassical QFT, and BH thermodynamics, reduces the problem to a \emph{single} universal one-loop constant \(\gamma_{\mathrm{rob}}\) associated with a real--Robin boundary on the replica cone. The closure leg fixes this constant from minimal, physical information principles that (i) calibrate entropy in bits per area and (ii) enforce lossless, reversible geometric metadata at curvature-threshold events. The outcome,
\[
\T=\frac{1}{4\ln 2},\qquad \Ical(\LL)=48\ln 2,\qquad E_{\text{slab}}=M,
\]
contains no adjustable parameters and is consistent with all orthodox checks. Phenomenologically, the result predicts a fixed interior radius \(r_{\LL}=R_{S}/\sqrt{4\ln 2}\), exact area increments \(\Delta A=4\ln 2\,\lp^{2}\) per irreversible bit, and mass-as-slab-energy---all of which can be connected to ringdown phases, horizon thermodynamics, and potential higher-dimensional extensions.

\section*{Acknowledgments and notes}
This note is self-contained. The real--Robin heat-kernel coefficients, quasilocal mass flux relations, and replica localization on entangling surfaces are standard; explicit values (e.g.\ for \(c_{\text{bulk}}\) and \(c_{\text{bdry}}^{(\text{Robin})}\)) can be taken from canonical heat-kernel references for manifolds with boundary. The closure step uses only the bit-normalized BH area law and a minimal reversible-record postulate to enforce losslessness.

% ================== APPENDIX: KERR CONSISTENCY CHECKS ==================
\appendix
\section[Kerr consistency checks]{Kerr consistency checks for \(\T=\tfrac{1}{4\ln 2}\)}

\subsection*{Kerr background and local curvature ruler}

Consider the Kerr spacetime with mass \(M\) and specific angular momentum \(a\) in Boyer–Lindquist coordinates \((t,r,\theta,\phi)\).
Let \(r_\pm=M\pm\sqrt{M^{2}-a^{2}}\) be the horizon radii and \(r_+\) the outer (event) horizon.
The horizon area and surface gravity are
\[
\AH=4\pi\bigl(r_+^{2}+a^{2}\bigr),\qquad
\kappa_{\mathrm H}=\frac{r_+-r_-}{2\,(r_+^{2}+a^{2})},\qquad
\Omega_{\mathrm H}=\frac{a}{r_+^{2}+a^{2}}.
\]

The Kretschmann scalar is (standard, see e.g. textbooks)
\[
K(r,\theta)=\frac{48M^{2}\,\bigl(r^{6}-15a^{2}r^{4}\cos^{2}\theta+15a^{4}r^{2}\cos^{4}\theta-a^{6}\cos^{6}\theta\bigr)}
{\bigl(r^{2}+a^{2}\cos^{2}\theta\bigr)^{6}}.
\]

In Schwarzschild we used the local, dimensionless ruler \(\Ical(r)\equiv K(r)\,r^{4}\), which is constant on round spheres.
In Kerr, axial symmetry breaks spherical symmetry, so any strictly \emph{global} function \(\Ical(r,\theta)\) varies with \(\theta\).
However, the \emph{QES stationarity} used in the main text is \emph{local}: it is imposed on each tile of a smooth spacelike 2–surface.
Hence the correct generalization is the \emph{pointwise} condition
\[
\boxed{\quad \Ical_{\text{loc}}(p)\;\equiv\;K\bigl(p\bigr)\,\Rcal(p)^{4}\;=\;48\,\ln 2\quad\text{for all points }p\in\LL\ ,\quad}
\]
where \(\Rcal(p)\) is the \emph{local areal radius} of the tile (defined from the induced 2–metric \(h_{ab}\) via \(\delta A=4\pi\,\Rcal^{2}\,\delta\Omega\) for an infinitesimal solid angle \(\delta\Omega\)).
This reduces to \(I=K r^{4}\) in Schwarzschild (\(\Rcal=r\)) and ensures the threshold is \emph{intrinsically} defined.

\paragraph{Why this is the right object.}
In the local Rindler patch anchored to \(\LL\), the QES equation reads
\[
\Theta_{\text{gen}}=\theta+4\ell_{P}^{2}\,\partial_{\lambda}\!\Big(\frac{S_{\mathrm{out}}}{\delta A}\Big)=0.
\]
Both terms carry the same local length scale \(\ell_\perp^{-1}\) set by the tile (\(\theta\sim-2/\ell_\perp\)).
The QFT response with a lossless (real–Robin) inner boundary is
\[
\partial_{\lambda}\!\Big(\tfrac{S_{\mathrm{out}}}{\delta A}\Big)=\frac{\gamma_{\mathrm{rob}}}{\ell_\perp}\;\frac{K\,\Rcal^{4}}{12}\ .
\]
The common factor \(1/\ell_\perp\) cancels, leaving the purely \emph{local} threshold \(K\,\Rcal^{4}=48\ln 2\).
Thus, the same number \(48\ln 2\) controls the selection in Kerr, pointwise on \(\LL\).

\subsection*{Existence and uniqueness of \texorpdfstring{\(\LL\)}{L} in Kerr (local statement)}

Let \(\LL\) be a smooth, axisymmetric 2–surface inside the horizon, described in BL coordinates as \(r=r_{\LL}(\theta)\) on a constant-\(t\) slice.
Near any point \(p\in\LL\) choose Gaussian normal coordinates adapted to \(\LL\) so that the analysis of Sec.~3 applies verbatim.
Because \(\gamma_{\mathrm{rob}}\) is universal (lossless, bits), the stationarity fixes at every \(p\)
\[
K\bigl(r_{\LL}(\theta),\theta\bigr)\,\Rcal\bigl(r_{\LL}(\theta),\theta\bigr)^{4} = 48\ln 2,
\]
which determines \(r_{\LL}(\theta)\) uniquely given regularity and axial symmetry.
Hence \(\LL\) exists and is unique (within the smooth axisymmetric class) and reduces to a round sphere when \(a\to 0\).

\subsection*{Small-spin check: explicit expansion to \texorpdfstring{\(\mathcal{O}(a^{2})\)}{O}}

To make the statement concrete, we expand the true local invariant, \(\Ical_{\text{loc}}(p) = K(p)\,\Rcal(p)^{4}\), for a small spin parameter \(a/M\ll 1\). The physically appropriate length scale is the local areal radius, which to this order is given by \(\Rcal(p)^4 = (r^2+a^2\cos^2\theta)^2 \approx r^4 + 2r^2a^2\cos^2\theta\). Using the standard expansion for the Kretschmann scalar in Kerr,
\[
K(r,\theta) = \frac{48M^2}{r^6} - \frac{720 M^2 a^2 \cos^2\theta}{r^8} + \frac{288 M^2 a^2}{r^8} + \mathcal{O}(a^4),
\]
the local invariant becomes:
\[
\Ical_{\text{loc}}(r,\theta) = K(r,\theta)\,\Rcal(r,\theta)^4 = \frac{48M^2}{r^2} + a^2\frac{M^2}{r^4}\left(384 - 816\cos^2\theta\right) + \mathcal{O}(a^4).
\]
We can evaluate this at the horizon radius, \(r_+ = 2M - \frac{a^2}{2M} + \mathcal{O}(a^4)\), to find the invariant's value there:
\[
\Ical_{\mathrm H}(\theta) \equiv \Ical_{\text{loc}}(r_+,\theta) = 12 + \frac{a^2}{M^2}\left(18 - \frac{117}{2}\cos^2\theta\right) + \mathcal{O}(a^4).
\]

Now, we posit the ledger surface \(r_{\LL}(\theta)\) is a small deformation from the Schwarzschild solution, \(r_0 = R_S\sqrt{\alpha}\), where \(\alpha=1/(4\ln 2)\) is the value derived in the main text.
\[
r_{\LL}(\theta) = r_0 + \delta r(\theta),\qquad r_0 = 2M\sqrt{\alpha}, \qquad \delta r = \mathcal{O}(a^2).
\]
We impose the pointwise threshold \(\Ical_{\text{loc}}(r_{\LL}(\theta),\theta) = 48\ln 2 = 12/\alpha\). Expanding the invariant around \(r_0\) gives:
\[
\Ical_{\text{loc}}(r_{\LL}) \approx \frac{48M^2}{r_0^2}\left(1 - \frac{2\delta r}{r_0}\right) + a^2\frac{M^2}{r_0^4}\left(384 - 816\cos^2\theta\right) = \frac{12}{\alpha}.
\]
Since \(48M^2/r_0^2 = 12/\alpha\), this simplifies, allowing us to solve for the shape correction \(\delta r(\theta)\) that ensures the invariant is constant on the ledger surface:
\[
\delta r(\theta) = \frac{a^2}{M\sqrt{\alpha}}\left(4 - \frac{17}{2}\cos^2\theta\right) + \mathcal{O}(a^4).
\]
This expression for the ledger's shape is the correct one derived from the physically-motivated local invariant \(\Ical_{\text{loc}}\). It has the expected structure of an isotropic shift plus a quadrupole \(`\cos^2\theta`\) deformation.

\paragraph{Area fraction check.}
Compute the area of the ledger surface \(\LL\) to \(\mathcal{O}(a^{2})\). Using the corrected shape deformation \(\delta r(\theta)\), a perturbative calculation of the surface area in the Kerr metric yields a small deviation from the constant fraction \(\alpha\). The area fraction is found to be:
\[
\boxed{\quad \frac{\AL}{\AH}=\alpha + \frac{a^2}{M^2}\left(\frac{3\alpha+8}{12}\right) + \mathcal{O}(a^{4}),\qquad \left(\alpha=\frac{1}{4\ln 2}\right)\quad}
\]
In words: there exists a smooth axisymmetric surface \(\LL\) that obeys the \emph{local} curvature threshold everywhere. For small spin, this requirement fixes the surface's shape, which in turn determines a specific, calculable correction to the \emph{global} area fraction at second order in \(a\).

\subsection*{First law, losslessness, and energy bookkeeping in Kerr}

The Killing generator of the Kerr horizon is \(\chi^{a}=t^{a}+\Omega_{\mathrm H}\,\phi^{a}\).
With a \emph{lossless} (real–Robin) inner surface \(\LL\) that is everywhere orthogonal to the ingoing null congruence and co-rotates with \(\Omega_{\mathrm H}\), the QES stationarity is imposed in the \emph{local} Rindler frame of \(\chi^{a}\).
The exterior evolution remains unitary, with no flux through \(\LL\) in the \(\chi^{a}\)-frame.

The first law reads \cite{Bardeen1973}
\[
\delta M=\frac{\kappa_{\mathrm H}}{8\pi G}\,\delta \Acal_{\mathrm H}+\Omega_{\mathrm H}\,\delta J,
\]
and our area-fraction statement extends to
\[
\delta \AL=\alpha\,\delta \Acal_{\mathrm H}\qquad(\text{same }\alpha).
\]
Energy accounting in the slab between \(\LL\) and the horizon can be written in terms of the \(\chi^{a}\)-canonical energy:
for quasi-stationary processes,
\[
\frac{d}{dv}\bigl[M-\Omega_{\mathrm H} J\bigr]\;=\;\int_{\mathrm{EH}}T_{ab}\,\chi^{a}\chi^{b}\,\dd\mathcal{A},
\qquad
\frac{d}{dv}\bigl[\mathcal{E}_{\text{in}}(\LL)\bigr]=0,
\]
so the slab accounts for the full change of the \(\chi\)-energy, exactly as in Schwarzschild with \(J=0\).
In the stationary limit the “slab energy” reproduces the Komar/Smarr relation consistently with our fixed \(\alpha\) and lossless \(\LL\).

\subsection*{Summary of Kerr consistency}

\begin{itemize}[leftmargin=1.4em]
\item \emph{Local threshold:} the QES stationarity with a lossless boundary is pointwise and yields the same number
\[
K\,\Rcal^{4}=48\ln 2 \quad \text{everywhere on }\LL.
\]
\item \emph{Existence/uniqueness:} for each \(a\) there is a smooth axisymmetric \(\LL\) inside the horizon, uniquely determined by the local condition (up to diffeos).
\item \emph{Area fraction:} \(\AL/\AH=\alpha\) with \(\alpha=\tfrac{1}{4\ln 2}\) holds to \(\mathcal{O}(a^{2})\) explicitly; higher orders follow by the same local–global matching \cite{DCTQG}.
\item \emph{First law/energy:} the lossless condition uses the co-rotating \(\chi^{a}\); the first law and Smarr relations remain consistent with the fixed \(\alpha\) and with our “no flux through \(\LL\)” assumption.
\end{itemize}

\paragraph{Remark on invariants.}
Throughout we retained the same scalar \(K\) and paired it with the \emph{intrinsic} tile radius \(\Rcal\) so that \(K\,\Rcal^{4}\) is a genuine local, gauge-invariant trigger.
In the \(a\to 0\) limit this reduces to \(\Ical(r)=K(r)\,r^{4}\) with \(\Ical(\LL)=48\ln 2\), as in the main text.

\bigskip
\noindent\emph{Conclusion.} The Kerr checks support the same universal value \(\T =1/(4\ln 2)\):
rotation deforms the \(\LL\) surface slightly but neither shifts the \emph{local} threshold \(K\,\Rcal^{4}=48\ln 2\) nor the \emph{global} area fraction \(\AL/\AH=\alpha\).

\include{papers/p01_trigger/appendix}

\bibliographystyle{unsrt}
\bibliography{bibliography/dct_refs, bibliography/external_refs}

\end{document}
